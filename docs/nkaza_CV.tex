%!TEX TS-program = xelatex
%!TEX encoding = UTF-8 Unicode
% Awesome CV LaTeX Template for CV/Resume
%
% This template has been downloaded from:
% https://github.com/posquit0/Awesome-CV
%
% Author:
% Claud D. Park <posquit0.bj@gmail.com>
% http://www.posquit0.com
%
%
% Adapted to be an Rmarkdown template by Mitchell O'Hara-Wild
% 23 November 2018
%
% Template license:
% CC BY-SA 4.0 (https://creativecommons.org/licenses/by-sa/4.0/)
%
%-------------------------------------------------------------------------------
% CONFIGURATIONS
%-------------------------------------------------------------------------------
% A4 paper size by default, use 'letterpaper' for US letter
\documentclass[11pt,a4paper,]{awesome-cv}

% Configure page margins with geometry
\usepackage{geometry}
\geometry{left=1.4cm, top=.8cm, right=1.4cm, bottom=1.8cm, footskip=.5cm}


% Specify the location of the included fonts
\fontdir[fonts/]

% Color for highlights
% Awesome Colors: awesome-emerald, awesome-skyblue, awesome-red, awesome-pink, awesome-orange
%                 awesome-nephritis, awesome-concrete, awesome-darknight

\definecolor{awesome}{HTML}{414141}

% Colors for text
% Uncomment if you would like to specify your own color
% \definecolor{darktext}{HTML}{414141}
% \definecolor{text}{HTML}{333333}
% \definecolor{graytext}{HTML}{5D5D5D}
% \definecolor{lighttext}{HTML}{999999}

% Set false if you don't want to highlight section with awesome color
\setbool{acvSectionColorHighlight}{true}

% If you would like to change the social information separator from a pipe (|) to something else
\renewcommand{\acvHeaderSocialSep}{\quad\textbar\quad}

\def\endfirstpage{\newpage}

%-------------------------------------------------------------------------------
%	PERSONAL INFORMATION
%	Comment any of the lines below if they are not required
%-------------------------------------------------------------------------------
% Available options: circle|rectangle,edge/noedge,left/right

\name{Nikhil}{Kaza}

\address{314 New East, Chapel Hill, NC 27599-3140}

\mobile{+1 919 962 4767}
\email{\href{mailto:nkaza@unc.edu}{\nolinkurl{nkaza@unc.edu}}}
\homepage{nkaza.github.io}
\github{nkaza}

% \gitlab{gitlab-id}
% \stackoverflow{SO-id}{SO-name}
% \skype{skype-id}
% \reddit{reddit-id}


\usepackage{booktabs}

\providecommand{\tightlist}{%
	\setlength{\itemsep}{0pt}\setlength{\parskip}{0pt}}

%------------------------------------------------------------------------------


\usepackage{xurl}

% Pandoc CSL macros
% definitions for citeproc citations
\NewDocumentCommand\citeproctext{}{}
\NewDocumentCommand\citeproc{mm}{%
  \begingroup\def\citeproctext{#2}\cite{#1}\endgroup}
\makeatletter
 % allow citations to break across lines
 \let\@cite@ofmt\@firstofone
 % avoid brackets around text for \cite:
 \def\@biblabel#1{}
 \def\@cite#1#2{{#1\if@tempswa , #2\fi}}
\makeatother
\newlength{\cslhangindent}
\setlength{\cslhangindent}{1.5em}
\newlength{\csllabelwidth}
\setlength{\csllabelwidth}{3em}
\newenvironment{CSLReferences}[2] % #1 hanging-indent, #2 entry-spacing
 {\begin{list}{}{%
  \setlength{\itemindent}{0pt}
  \setlength{\leftmargin}{0pt}
  \setlength{\parsep}{0pt}
  % turn on hanging indent if param 1 is 1
  \ifodd #1
   \setlength{\leftmargin}{\cslhangindent}
   \setlength{\itemindent}{-1\cslhangindent}
  \fi
  % set entry spacing
  \setlength{\itemsep}{#2\baselineskip}}}
 {\end{list}}
\usepackage{calc}
\newcommand{\CSLBlock}[1]{\hfill\break\parbox[t]{\linewidth}{\strut\ignorespaces#1\strut}}
\newcommand{\CSLLeftMargin}[1]{\parbox[t]{\csllabelwidth}{\strut#1\strut}}
\newcommand{\CSLRightInline}[1]{\parbox[t]{\linewidth - \csllabelwidth}{\strut#1\strut}}
\newcommand{\CSLIndent}[1]{\hspace{\cslhangindent}#1}

\begin{document}

% Print the header with above personal informations
% Give optional argument to change alignment(C: center, L: left, R: right)
\makecvheader

% Print the footer with 3 arguments(<left>, <center>, <right>)
% Leave any of these blank if they are not needed
% 2019-02-14 Chris Umphlett - add flexibility to the document name in footer, rather than have it be static Curriculum Vitae
\makecvfooter
  {June 03, 2024}
    {Nikhil Kaza~~~·~~~Curriculum Vitae}
  {\thepage~ of \pageref{LastPage}~}


%-------------------------------------------------------------------------------
%	CV/RESUME CONTENT
%	Each section is imported separately, open each file in turn to modify content
%------------------------------------------------------------------------------



\section{Education}\label{education}

\begin{cventries}
    \cventry{Ph.D. in Regional Planning}{University of Illinois}{Urbana Champaign}{2008}{}\vspace{-4.0mm}
    \cventry{M.S. in Applied Mathematics}{University of Illinois}{Urbana Champaign}{2007}{}\vspace{-4.0mm}
    \cventry{M.U. P. in Urban Planning}{University of Illinois}{Urbana Champaign}{2004}{}\vspace{-4.0mm}
    \cventry{B.Arch (Hons.) in Architecture}{Indian Institute of Technology}{Kharagpur, India}{2001}{}\vspace{-4.0mm}
\end{cventries}

\section{Experience}\label{experience}

\begin{cventries}
    \cventry{Professor}{Department of City \& Regional Planning}{University of North Carolina at Chapel Hill}{Jul 2021 - Present}{}\vspace{-4.0mm}
    \cventry{Director}{Center for Urban \& Regional Studies}{University of North Carolina at Chapel Hill}{Jul 2021 -Jul 2024}{}\vspace{-4.0mm}
    \cventry{Adjunct Associate Professor}{Environment, Ecology \& Energy Program}{University of North Carolina at Chapel Hill}{Jul 2016 - Jul 2024}{}\vspace{-4.0mm}
    \cventry{Host Professor}{School of Architecture, Civil \& Environmental Engineering}{Ecole Polytechnique Fédérale de Lausane}{Jan 2019 - May 2019}{}\vspace{-4.0mm}
    \cventry{Associate Professor}{Department of City \& Regional Planning}{University of North Carolina at Chapel Hill}{Jul 2015  - Jul 2021}{}\vspace{-4.0mm}
    \cventry{Associate Chair}{Department of City \& Regional Planning}{University of North Carolina at Chapel Hill}{Jul 2015  - Jul 2017}{}\vspace{-4.0mm}
    \cventry{Assistant Professor}{Department of City \& Regional Planning}{University of North Carolina at Chapel Hill}{Jul 2009  - Jul 2015}{}\vspace{-4.0mm}
    \cventry{Adjunct Assistant Professor}{Curriculum of Ecology \& Environment}{University of North Carolina at Chapel Hill}{Jan 2012 - Jul 2016}{}\vspace{-4.0mm}
    \cventry{Sustainability Systems Modeler}{EDAW Inc.}{Washington, DC}{Jan 2008 - Dec 2008}{}\vspace{-4.0mm}
    \cventry{Post Doctoral Fellow}{National Center for Smart Growth Research \& Education}{University of Maryland at College Park}{Nov 2007 - Jun 2009}{}\vspace{-4.0mm}
\end{cventries}

\section{Publications}\label{publications}

\subsection{Refereed Journal Papers}\label{refereed-journal-papers}

Total number of articles in referred journals: \textbf{43}

\phantomsection\label{refs-92333bf2587d2e2928ed3fc723ae37ba}
\begin{CSLReferences}{1}{0}
\bibitem[\citeproctext]{ref-Branham:2022te}
Branham, J., Salvesen, D., Kaza, N., \& K. BenDor, T. (2024). A wrench
in the machine: How subsidy removal alters the politics of coastal
development. \emph{Journal of the American Planning Association},
\emph{90}(1), 18--29.
\url{https://doi.org/10.1080/01944363.2022.2119156}

\bibitem[\citeproctext]{ref-Hino:2023uq}
Hino, M., BenDor, T. K., Branham, J., Kaza, N., Sebastian, A., \&
Sweeney, S. (2024). Growing safely or building risk? \emph{Journal of
the American Planning Association}, \emph{90}(1), 1--11.
\url{https://doi.org/10.1080/01944363.2022.2141821}

\bibitem[\citeproctext]{ref-kazaMulticlassCompactnessIndex2024}
Kaza, N. (2024). Multiclass compactness index for urban areas. \emph{The
Professional Geographer}, 1--11.
\url{https://doi.org/10.1080/00330124.2024.2328689}

\bibitem[\citeproctext]{ref-dicarloCustomerComplaintManagement2023a}
DiCarlo, M., Berglund, E. Z., Kaza, N., Grieshop, A., Shealy, L., \&
Behr, A. (2023). Customer complaint management and smart technology
adoption by community water systems. \emph{Utilities Policy}, \emph{80},
101465. \url{https://doi.org/10.1016/j.jup.2022.101465}

\bibitem[\citeproctext]{ref-LI:2018}
Li, C., Song, Y., Kaza, N., \& Burghardt, R. (2023). Explaining spatial
variations in residential energy usage intensity in chicago: The role of
urban form and geomorphometry. \emph{Journal of Planning Education and
Research}, \emph{43}(2), 317--331.
https://doi.org/\url{https://doi.org/10.1016/j.enbuild.2017.10.007}

\bibitem[\citeproctext]{ref-branhamRemovingFederalSubsidies2021}
Branham, J., Kaza, N., BenDor, T., Salvesen, D., \& Onda, K. (2022).
Removing federal subsidies from high-hazard coastal areas slows
development. \emph{Frontiers in Ecology \& Environment}, \emph{20}(9),
500--506.

\bibitem[\citeproctext]{ref-Kaza:2019ab}
Kaza, N. (2022). Landscape shape adjusted compactness index for urban
areas. \emph{Geoforum}, \emph{87}(1399-1409).
\url{https://doi.org/10.1007/s10708-020-10262-9}

\bibitem[\citeproctext]{ref-wangSocioeconomicDisparitiesActivitytravel2022}
Wang, J., Kaza, N., McDonald, N. C., \& Khanal, K. (2022).
Socio-economic disparities in activity-travel behavior adaptation during
the COVID-19 pandemic in North Carolina. \emph{Transport Policy},
\emph{125}, 70--78. \url{https://doi.org/10.1016/j.tranpol.2022.05.012}

\bibitem[\citeproctext]{ref-Branham:2021CB}
Branham, J., Onda, K., Kaza, N., BenDor, T. K., \& Salvesen, D. (2021).
How does the removal of federal subsidies affect investment in coastal
protection infrastructure? \emph{Landuse Policy}, \emph{102}, 105245.

\bibitem[\citeproctext]{ref-Kaza:2019aa}
Kaza, N., \& Nesse, K. (2021). Characterizing the regional structure in
united states: A county-based analysis of labor market centrality.
\emph{International Journal of Regional Science Review}, \emph{44}(5),
487--614. https://doi.org/\url{https://doi.org/10.1177/0160017620946082}

\bibitem[\citeproctext]{ref-He:2019aa}
He, M., Glasser, J., Pritchard, N., Bhamidi, S., \& Kaza, N. (2020).
Demarcating regions using community detection in commuting networks.
\emph{PLOS One}, \emph{15}(4), e0230941.

\bibitem[\citeproctext]{ref-Kaza:2019transport}
Kaza, N. (2020). Urban form and transportation energy consumption.
\emph{Energy Policy}, \emph{136}, 111049.
https://doi.org/\url{https://doi.org/10.1016/j.enpol.2019.111049}

\bibitem[\citeproctext]{ref-Onda:2019aa}
Onda, K., Branham, J., BenDor, T. K., Kaza, N., \& Salvesen, D. (2020).
Does removal of federal subsidies discourage development? An evaluation
of the US coastal barrier resources act. \emph{PLOS One}, \emph{15}(6),
e0233888.
https://doi.org/\url{https://doi.org/10.1371/journal.pone.0233888}

\bibitem[\citeproctext]{ref-Peng:2020aa}
Peng, K., \& Kaza, N. (2020). Association between neighborhood food
access, household income, and purchase of snacks and beverages in the
united states. \emph{International Journal of Environmental Research and
Public Health}, \emph{17}, 7517.

\bibitem[\citeproctext]{ref-pesantez_berglund_kaza}
Pesantez, J. E., Berglund, E. Z., \& Kaza, N. (2020). Smart meters data
for modeling and forecasting water demand at the user-level.
\emph{Environmental Modelling and Software}, \emph{125}, 104633.
https://doi.org/\url{https://doi.org/10.1016/j.envsoft.2020.104633}

\bibitem[\citeproctext]{ref-kaza:vainforesight}
Kaza, N. (2019). Vain foresight: Against the idea of implementation in
planning. \emph{Planning Theory}, \emph{18}(4), 410--428.

\bibitem[\citeproctext]{ref-Kaza:Magpie_India}
Onda, K., Sinha, P., Stevens, F., Gaughan, A. E., \& Kaza, N. (2019).
Missing millions: Undercounting urbanisation in india. \emph{Population
\& Environment}, \emph{41}(2), 126--150.
\url{https://doi.org/10.1007/s11111-019-00329-2}

\bibitem[\citeproctext]{ref-PengKaza:fruits}
Peng, K., \& Kaza, N. (2019). Built environment and the purchase of
fruits and vegetables in united states households. \emph{Public Health
Nutrition}, \emph{22}(13), 2436--2447.
\url{https://doi.org/10.1017/S1368980019000910}

\bibitem[\citeproctext]{ref-LI:2017}
Li, C., Song, Y., \& Kaza, N. (2018). Urban form and household
electricity consumption: A multilevel study. \emph{Energy and
Buildings}, \emph{158}(1), 181--193.
\url{https://doi.org/10.1016/j.enbuild.2017.10.007}

\bibitem[\citeproctext]{ref-Qiu:2017jk}
Qiu, S., \& Kaza, N. (2017). Evaluating the impacts of the clean cities
program. \emph{Science of The Total Environment}, \emph{579}, 254--262.
\url{https://doi.org/10.1016/j.scitotenv.2016.11.119}

\bibitem[\citeproctext]{ref-Kaza:EDQ}
Hartley, D. A., Kaza, N., \& Lester, T. W. (2016). Are america's inner
cities competitive? Evidence from the 2000s. \emph{Economic Development
Quarterly}, \emph{30}(2).
\url{http://dx.doi.org/10.1177/0891242416638932}

\bibitem[\citeproctext]{ref-Kaza:HPD:2016}
Kaza, N., Riley, S., Quercia, R. G., \& Tian, C. (2016). Location
efficiency \& mortgage risks for low-income households. \emph{Housing
Policy Debate}, \emph{26}(4-5), 750--765.
\url{http://dx.doi.org/10.1080/10511482.2016.1159972}

\bibitem[\citeproctext]{ref-Kaza:TDA}
Kaza, N. (2015). Time dependent accessibility. \emph{Journal of Urban
Management}, \emph{4}(1), 24--39.
\url{http://dx.doi.org/10.1016/j.jum.2015.06.001}

\bibitem[\citeproctext]{ref-Kaza:wp:yo}
McCarty, J., \& Kaza, N. (2015). Urban form and air quality.
\emph{Landscape and Urban Planning}, \emph{139}, 168--179.
\url{http://dx.doi.org/10.1016/j.landurbplan.2015.03.008}

\bibitem[\citeproctext]{ref-Zapata:dz}
Zapata, M., \& Kaza, N. (2015). Radical uncertainty: Scenario planning
for futures. \emph{Environment and Planning B: Planning and Design},
\emph{42}(4), 754--770.
\url{http://www.envplan.com/epb/fulltext/bforth/b39059.pdf}

\bibitem[\citeproctext]{ref-Kaza:2013fk}
Kaza, N. (2014). Persons, polities and planning. \emph{Planning Theory},
\emph{13}(2), 136--151. \url{http://dx.doi.org/10.1177/1473095213490687}

\bibitem[\citeproctext]{ref-Kaza:2013fv}
Kaza, N., \& Patane, M. (2014). The land use energy connection.
\emph{Journal of Planning Literature}, \emph{29}(4), 355--369.
\url{http://dx.doi.org/10.1177/0885412214542049}

\bibitem[\citeproctext]{ref-Kaza:2013tg}
Kaza, N., Tian, C., \& Quercia, R. G. (2014). Home energy efficiency and
mortgage risks. \emph{Cityscape}, \emph{16}(1), 279--298.
\url{http://www.huduser.org/portal/periodicals/cityscpe/vol16num1/ch16.pdf}

\bibitem[\citeproctext]{ref-Brookshire:2013bs}
Brookshire, D., \& Kaza, N. (2013). Planning for seven generations:
Energy planning of american indian tribes. \emph{Energy Policy},
\emph{62}, 1506--1514.
\url{http://dx.doi.org/10.1016/j.enpol.2013.07.021}

\bibitem[\citeproctext]{ref-kaza_changing_2013}
Kaza, N. (2013). The changing urban landscape of continental united
states. \emph{Landscape and Urban Planning}, \emph{110}, 74--86.
\url{http://dx.doi.org/10.1016/j.landurbplan.2012.10.015}

\bibitem[\citeproctext]{ref-Kaza:2013kx}
Kaza, N., \& BenDor, T. K. (2013). Land value impacts of aquatic
ecosystem restoration. \emph{Journal of Environmental Management},
\emph{127}, 289--299.
\url{http://dx.doi.org./10.1016/j.jenvman.2013.04.047}

\bibitem[\citeproctext]{ref-Kaza:2013uq}
Kaza, N., Lester, T. W., \& Rodriguez, D. (2013). The spatio-temporal
clustering of green buildings in the US. \emph{Urban Studies},
\emph{50}(16), 3262--3282.
\url{http://dx.doi.org/10.1177/0042098013484540}

\bibitem[\citeproctext]{ref-Lester:wp:jo}
Lester, T. W., Kaza, N., \& Kirk, S. (2013). Making room for
manufacturing: Understanding industrial land conversion in cities.
\emph{Journal of American Planning Association}, \emph{79}(4), 295--313.
\url{http://dx.doi.org/10.1080/01944363.2014.915369}

\bibitem[\citeproctext]{ref-BenDor:2028vn}
BenDor, T. K., \& Kaza, N. (2012). A theory of spatial systems
archetypes. \emph{System Dynamics Review}, \emph{28}(2), 109--130.
\url{http://dx.doi.org/10.1002/sdr.1470}

\bibitem[\citeproctext]{ref-Kaza:2012zr}
Kaza, N., \& Hopkins, L. D. (2012). Intentions, urban plans and
information systems. \emph{International Journal of Geographic
Information Science}, \emph{26}(3), 557--576.
\url{http://dx.doi.org/10.1080/13658816.2011.603337}

\bibitem[\citeproctext]{ref-Chakraborty:2011lr}
Chakraborty, A., Kaza, N., Knaap, G. J., \& Deal, B. (2011). Robust
plans and contingent plans: Scenario planning for an uncertain world.
\emph{Journal of the American Planning Association}, \emph{77}(3),
1--17. \url{http://dx.doi.org/10.1080/01944363.2011.582394}

\bibitem[\citeproctext]{ref-Kaza:2011dq}
Kaza, N., Knaap, I., Knaap, G. J., \& Lewis, R. (2011). Peak oil, urban
form and public health: Exploring the connections. \emph{American
Journal of Public Health}, \emph{101}(9), 1598--1606.
\url{http://dx.doi.org/10.2105/AJPH.2011.300192}

\bibitem[\citeproctext]{ref-Kaza:2011qf}
Kaza, N., Towe, C., \& Ye, X. (2011). A hybrid land conversion model
incorporating multiple end uses. \emph{Agricultural and Resource
Economics Review}, \emph{40}(3), 341--359.
\href{http://ageconsearch.umn.edu/bitstream/120447/2/kaza\%20-\%20current.pdf}{http://ageconsearch.umn.edu/bitstream/120447/2/kaza
- current.pdf}

\bibitem[\citeproctext]{ref-Kaza:2010lr}
Kaza, N. (2010). Understanding the spectrum of residential energy
consumption: A quantile regression approach. \emph{Energy Policy},
\emph{38}(11), 6574--6585.
\url{http://dx.doi.org/10.1016/j.enpol.2010.06.028}

\bibitem[\citeproctext]{ref-Kaza:2009nx}
Kaza, N., Finn, D., \& Hopkins, L. D. (2009). Updating plans: A
historiography of decisions over time. \emph{Journal of Information
Technology in Construction}, \emph{15}, 159--168.
\url{http://www.itcon.org./data/works/att/2010_13.content.03108.pdf}

\bibitem[\citeproctext]{ref-Kaza:2009fk}
Kaza, N., \& Hopkins, L. D. (2009). In what circumstances should plans
be public? \emph{Journal of Planning Education and Research},
\emph{28}(4), 491--502. \url{http://dx.doi.org/10.1177/0739456X08330978}

\bibitem[\citeproctext]{ref-Kaza:2006oq}
Kaza, N. (2006). Tyranny of the median: A reflection on the
participatory urban processes. \emph{Planning Theory}, \emph{5}(3),
255--270. \url{http://dx.doi.org/10.1177/1473095206068630}

\bibitem[\citeproctext]{ref-hopkins_representing_2005}
Hopkins, L. D., Kaza, N., \& Pallathucheril, V. G. (2005). Representing
urban development plans and regulations as data: A planning data model.
\emph{Environment \& Planning B : Planning and Design}, \emph{32}(4),
597--615. \url{http://envplan.com/epb/fulltext/b32/b31178.pdf}

\end{CSLReferences}

\subsection{Manuscripts}\label{manuscripts}

Manuscripts in progress/submitted/accepted: \textbf{5}

\phantomsection\label{refs-562c2f672dbd7502ca65d02a8d811cb9}
\begin{CSLReferences}{1}{0}
\bibitem[\citeproctext]{ref-BranhamConservation}
Branham, J., Kaza, N., \& BenDor, T. (n.d.). Do development
disincentives influence land conservation activity? In \emph{Frontiers
of Ecology \& Environment}.

\bibitem[\citeproctext]{ref-DonaldDashboardNotDead}
Donald, B., Brail, S., Lowe, N., DeLoyde, C., Heatwole, K., Hernandez,
F., Hill-Tout, K., Kaza, N., Khanal, K., Planey, D., \& Wang, J. (n.d.).
The Dashboard is not dead: Dashboards as effective tools in skills
building, sense-making and community collaboration. In \emph{Journal of
American Planning Association}.

\bibitem[\citeproctext]{ref-kazaetal2018whoseplan}
Kaza, N., Brookshire, D., Toledo, K., Perrit, M., \& Murphy, S. (n.d.).
Whose plan is it anyway? Energy planning by american indian tribes in
the united states. In \emph{Journal of Planning Education and Research}.

\bibitem[\citeproctext]{ref-KhanalEnergytransition}
Khanal, K., Lowe, N., \& Kaza, N. (n.d.). Retraining for energy
transition: A workforce development approach using occupational
similarity and unsupervised clustering. In \emph{Economic Development
Quarterly}.

\bibitem[\citeproctext]{ref-LesterKazaMcAdam:2018}
Lester, T. W., Kaza, N., \& McAdam, T. (n.d.). Splintered metropolitan
opportunity in the united states? Re-examining the 'mismatch' between
emerging employment centers and distressed neighborhoods. In
\emph{Economic Development Quarterly}.

\end{CSLReferences}

\subsection{Book Chapters}\label{book-chapters}

Total number of chapters in collected volumes: \textbf{4}

\phantomsection\label{refs-aaaf969735661a92bc15933de5718ba6}
\begin{CSLReferences}{1}{0}
\bibitem[\citeproctext]{ref-Becker:2020}
Becker, J., \& Kaza, N. (2022). Tale of two sprawls: Energy planning and
challenges for smart growth 2.0. In A. Chakraborty, R. Lewis, \& G. J.
Knaap (Eds.), \emph{Handbook on smart urban growth} (pp. 291--302).
Edward Elgar.

\bibitem[\citeproctext]{ref-kaza_knaap_2011}
Kaza, N., \& Knaap, G. J. (2011). Principles of planning for economists.
In N. Brooks, G. J. Knaap, \& K. P. Donaghy (Eds.), \emph{Oxford
handbook of urban economics and planning} (pp. 29--50). Oxford
University Press.
\url{http://dx.doi.org/10.1093/oxfordhb/9780195380620.013.0003}

\bibitem[\citeproctext]{ref-kaza_ontology_2007}
Kaza, N., \& Hopkins, L. D. (2007). Ontology for land development
decisions and plans. In J. Teller, J. Lee, \& C. Roussey (Eds.),
\emph{Ontologies for urban development} (pp. 47--59). Springer-Verlag.
\url{http://dx.doi.org/10.1007/978-3-540-71976-2_5}

\bibitem[\citeproctext]{ref-hopkins_data_2005}
Hopkins, L. D., Kaza, N., \& Pallathucheril, V. G. (2005). A data model
to incorporate plans and regulations in urban simulation models. In D.
G. Maguire, M. Batty, \& M. Goodchild (Eds.), \emph{GIS, spatial
analysis and modeling} (pp. 173--202). ESRI Press.

\end{CSLReferences}


\label{LastPage}~
\end{document}
